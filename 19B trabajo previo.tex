
\documentclass[article, 11pt]{report}
\usepackage{matlab-prettifier}
\usepackage[T1]{fontenc}
\usepackage{lmodern}
\usepackage{graphicx}
\usepackage{wrapfig}
\usepackage{color}
\usepackage{hyperref}
\usepackage{amsmath}
\usepackage{amsfonts}
\usepackage{epstopdf}
\usepackage[table]{xcolor}
\usepackage[spanish]{babel}
\usepackage{fullpage} % changes the margin
\usepackage{graphicx} 
\usepackage{enumitem} 
\usepackage{chngcntr}
\renewcommand{\thesection}{\arabic{section}} 
\renewcommand{\thesubsection}{\thesection.\arabic{subsection}}
\usepackage[table]{xcolor}% http://ctan.org/pkg/xcolor
\usepackage{subcaption}
\usepackage{vmargin}
\usepackage{hyperref}
\usepackage{float}
\usepackage{array}

\usepackage{matlab}
\usepackage{listings}
\definecolor{codegreen}{rgb}{0,0.6,0}
\definecolor{codegray}{rgb}{0.5,0.5,0.5}
\definecolor{codepurple}{rgb}{0.58,0,0.82}
\definecolor{backcolour}{rgb}{0.95,0.95,0.92}
\lstdefinestyle{mystyle}{
	backgroundcolor=\color{backcolour},   
	commentstyle=\color{codegreen},
	keywordstyle=\color{magenta},
	numberstyle=\tiny\color{codegray},
	stringstyle=\color{codepurple},
	basicstyle=\ttfamily\footnotesize,
	breakatwhitespace=false,         
	breaklines=true,                 
	captionpos=b,                    
	keepspaces=true,                 
	numbers=left,                    
	numbersep=5pt,                  
	showspaces=false,                
	showstringspaces=false,
	showtabs=false,                  
	tabsize=2
}

\lstset{style=mystyle}

















\setpapersize{A4}
\setmargins{2.5cm}       % margen izquierdo
{1.5cm}                        % margen superior
{16.75cm}                      % anchura del texto
{23.42cm}                    % altura del texto
{10pt}                           % altura de los encabezados
{1cm}                           % espacio entre el texto y los encabezados
{0pt}                             % altura del pie de página
{1cm}
\begin{document}
	\section{Trabajo Previo}
	
	
	 Se quiere demostrar que en las ecuaciones:


$$ \phi = \arctan(\dfrac{1}{2\pi \nu RC })$$ 
	 

$$ Z = \sqrt{R^2 +  (\dfrac{1}{2 \pi \nu C})^2}	$$

Cuando la frecuencia $\nu$ tiende a cero el valor de la impedancia $Z$ tiende a infinito y el valor del desfase tiende a $\pi/2$ mientras que si la frecuencia tiende a infinito la impedancia tiende al valor de la resistencia $R$ y el valor del angulo de desfase es 0.

Se presentan las graficas del modulo de la impedancia $|Z|$ y del valor del angulo $\phi$ frente a un rango de frecuencias: $[30,3000]Hz$ considerando $R=2100\Omega$ y $C= 10^-7 F$	



	


\begin{figure}[H]
	\centering
	\begin{minipage}{.5\textwidth}
		\centering
		\includegraphics[width=.95\linewidth]{Curva1}
	\end{minipage}%
	\begin{minipage}{.5\textwidth}
		\centering
		\includegraphics[width=.95\linewidth]{Curva2}

	\end{minipage}
\end{figure}

El código de Matlab usado para las gráficas:

\lstset{language= Matlab, breaklines=true, basicstyle=\footnotesize}
\begin{lstlisting}[ language=Matlab, basicstyle=\small, keywordstyle = \color{blue}]
	C=10^-7;
	R=2100;
	v=30:1:3000;
	
	for i=1:length(v)
	Z(i)= sqrt(R^2 + (1/(2*pi*v(i)*C))^2);
	phi(i) = atan(1/(2*pi*v(i)*R*C));
	end
	
	loglog(v,Z)
	title('Modulo de la Impedancia frente a frecuencia')
	xlabel('Frecuencia (Hz)')
	ylabel('Modulo de la impedancia (k$\Omega$),'Interpreter='latex')
	grid on
	
	semilogx(v,phi)
	title('Angulo de desfase frente a frecuencia')
	xlabel('Frecuencia (Hz)')
	ylabel('Angulo de desfase (rad)')
	
	grid on
\end{lstlisting}


\end{document}